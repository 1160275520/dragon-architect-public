\documentclass{sig-alternate}
\usepackage{url}
\usepackage{color}
\toappear{}

\newcommand{\da}{\emph{Dragon Architect}}
\newcommand{\todo}[1]{{\color{red} TODO: #1}}

\begin{document}
\title{Optimizing in a Novice Programming Environment}
\author{Eric Butler, Aaron Bauer}
\maketitle{}.

\section{Introduction}

For the past year, we have been developing \da, a programming system for novices, for the purpose of exploring research questions in computer science pedagogy and language usability for novices. Core to the design of our tool is to provide both a \emph{low floor} and \emph{high ceiling}, which is important for both supporting a wide range of skill levels, and engaging students over an extended progression. Part of our vision for students’ progression through \da{} is tackling programs of increasing size and sophistication. The system also includes several debugging tools, and we are interested in studying when and how various debugging tools are beneficial to novices. 

For example, our system supports debugging tool that allows users to move the program execution state forward and backward in time by dragging a slider bar. Thus, the language simulator needs to be able to quickly jump to different points of execution. In the presence of larger programs, however, the na\"{i}ve implementation of this slider bar, storing an array of all program states, quickly runs into scalability issues. Optimization of our runtime became necessary to support large programs efficiently. 

Fortunately, ample optimization opportunities exist. The design of \da{}’s language is intentionally very simple, to increase accessibility for novices. As a consequence, large programs are both very repetitive and deterministic. To try to improve performance on larger programs, we investigated and created two dynamic optimization techniques. One aggressively caches and reuses results for procedure calls and loop bodies, and the other computes a small number of \emph{checkpoints} instead of every state, resimulating from the nearest checkpoint when necessary. Our evaluation shows \todo{SOMEHTING PROBABLY?}

\section{Related Work}
In our brief search of the literature, the closest previous work we found dealt with navigating program execution as a timeline for both novices~\cite{ko2004designing} and professional web development~\cite{burg2013interactive}.

\section{Background}

We first discuss technical details of the language and runtime environment used in \da. The grammar is given by \todo{A GRAMMAR THING}. To summarize, execution in \da{} consists of running a simple imperative language to compute a list of commands for the dragon in the 3d world, and this list of commands is applied to compute a new world state. An example program and resulting world state is shown in Figure~\todo{asdfasdf}.

We elide a fully formal description of \da’s language semantics and instead briefly describe how programs behave. The \texttt{command} statements describe operations for the dragon to perform in the 3d world. Because our language currently does not have any conditionals, this list of commands does not depend on the world state, or even what the semantics of the world are. Thus, program execution can be thought of as two phases, one in which the list of commands is computed, and another in which the list of commands is applied to the world state to compute the final world state. This is summarized in Figure~\todo{FIGURE}.

The other statements in \da{} behave similarly to constructs in most other imperative programming languages, and we omit a full description for brevity. The only local context allowed is through procedure arguments, so the only identifiers in scope are local procedure parameters and the global procedures names.

The world state consists of the dragon’s current position and orientation, and the locations of any cubes that have been placed.  The behavior of each of the basic commands on the world state is given by \todo{SOME THING}. Each command is relative to the dragon’s current position. Commands are intended to represent ``atomic'' operations in the world, and they are the unit that counts as a single step for the purpose of the time slider bar. That is why, for example, the \texttt{forward} command takes no arguments and only allows moving one step at a time. In practice, users do not use the \texttt{Command} statement directly; it’s uses are hidden behind a standard library of procedures, such as \texttt{Forward(x:int)}, which uses a repeat loop to issue $x$ \texttt{forward} commands.

Our original na\"{i}ve implementation of the time slider bar works by actually storing every single world state over the course of the simulation. This allows the student to randomly access and view any state in the middle of the computation. Figure~\todo{timeslider} shows the time slider in action. Throughout the evaluation, we test two different representations for world states: a mutable hash-table and an immutable tree-map. The former has dramatically better performance when computing only a single state, but the latter is much faster and more memory-efficient for computing all states, because the hash-table must be copied every step. However, as the evaluation shows, these data structures have very different performance characteristics on our new techniques.

\section{Technique} 

In addition to the existing, na\"{i}ve implementation, we implemented two new approaches to simulating program in \da. The goal for all of these techniques is to support jumping to arbitrary point in execution (where each command is one step) and getting the world state for that point. The original implementation did this by simply keeping a copy of the world state for every single command executed, which was very simple but scales very poorly for large programs. For the purposes of this project, ``large'' programs mean those with more than 100,000 commands. The example castle program in Figure~\todo{figure} is over 200,000 commands, for example.

\subsection{Checkpointed Simulation}
The first approach we implemented was mostly intended as a baseline for comparison. Instead of storing literally every state, this approach stores a set of \emph{checkpoint} states along with the simulation context for that state. In order to jump to an arbitrary state, the simulator starts executing from the nearest checkpoint before the desired state and runs the simulation as normal to that state. We expected this approach to see a speed-up approximately proportional to the ratio of number of checkpoints to total number of states, since the na\"{i}ve implementation’s runtime was dominated by copying states.

\subsection{World Delta-Based Simulation}
Because we need to execute the program immediately, there is no benefit to doing static analysis. We instead do dynamic analysis. There is no input to the program, and therefore no non-determinism. Thus, we are in a sense trying to evaluate a very big constant expression.

\subsubsection{World State Detlas}

World state deltas consist of two components: (1) a delta on the position and direction of the dragon and (2) a delta on the cubes present in the world. Deltas are created from a sequence of commands, and are independent of a specific starting state. They simply represent the change to an arbitrary state that would result if that sequence of commands is applied. 

Taking the position delta first, a na\"{i}ve approach (and, indeed, our initial approach) would be to represent it as an absolute delta in terms of the x-, y-, and z-dimensions. This is incompatible, however, with the nature of our language. Our language does not have movement commands like \texttt{move N spaces in the x direction} that specify an absolute direction. Instead it has a \texttt{forward} command whose effect is relative to the dragon’s current heading. 

This means a position delta must be represented such that it can be relative to any given initial direction. Our solution is to represent a position delta as a distance parallel to the initial direction, and a distance perpendicular to the initial direction. When it comes time to apply the position delta to a concrete state, this information is combined with the dragon’s initial direction to generate the delta in absolute terms. This only accounts for the dragon’s position in the x-z plane. The two other pieces of the dragon’s state, its direction and y-position, are represented separately. 

Since the dragon can never be facing up or down, and moves in those directions with specific \texttt{up} and \texttt{down} commands, the change in y-position can be stored as an absolute delta. Change in the dragon’s direction can also be represented as a single integer. When computing a delta, we increment a counter for every right turn and decrement that counter for every left turn. It is thus easy to compute the dragon’s new direction when applying a delta.

The final wrinkle of position deltas is the restriction that the dragon cannot go below the ground ($y=0$). If it is told to go \texttt{down} when already at ground level, it simple stays where it is. The impact of this is that a delta can only be relied upon if it would not move the dragon below the ground at any point. Hence, we track the greatest lower extent for each delta and application of a delta is conditional on the dragon’s current y-position being higher than this extent. 

The delta on cubes is represented as a map from position deltas (as described above, though without tracking change in direction, as it is not needed) to cubes that should be added or removed. Applying this delta consists of processing each entry in the map, using the current dragon position and direction to compute an absolute position from the position delta, and then adding or remove a cube at that location, as appropriate.

One critical assumption about the world state that allow this technique to work is that a delta can always be computed from only a list of commands; a concrete world state is not required. This only holds because our language does not support conditionals: any block of code will behave similarly regardless of the world state. They can be straightforwardly mapped to any concrete world state by transforming the delta relative to the dragon. We are planning on (eventually) implementing language features to allow conditions on, e.g., whether a block exists, so we will have to generalize this technique to support such a program. As a worst case, the system can at least fall back to a non-delta implementation for segments with conditions but use detlas elsewhere.

\subsection{Caching Procedures and Loops}

The only two constructs in the language for repetition are procedures and definite loops, and we primarily target these constructs for optimization. The overall goal is, for each statement we wish to optimize, to compute a world state delta for that statement. When we next see the same statement, rather than running the simulator, we try to apply the cached delta to our concrete world state. We call these procedures and loops \emph{cachable statements}.

We can only reuse cachable statements if they have the same concrete environment. In the language of \da{}, there are no global variables nor lexical closures for procedures. Therefore, the concrete parameters given to a procedure fully describe the environment. Likewise, for loops, we can fully describe the surrounding environment with the concrete values of the parameters passed to the procedure in which the loop is contained.

Optimized evaluation of a statement proceeds as follows. If the statement is not cacheable, we use the normal evaluation method. If it \emph{is} cachable, we first compute the surrounding concrete environment. This may involve evaluating the arguments passed to the procedure, or looking at the current set of in-scope variables. The environment and the pointer to the statement form a unique key. We use this key to look up a world delta from the cache. If one does not exist, we create the world delta for this statement by executing it normally. We then attempt to apply the world delta to the current world state. If it fails, SCREW IT, JUST WRITE SOME PSEUDOCODE

\subsection{Jumping to Arbitrary Point of Execution}

describe the algroithm

\section{Evaluation}
made a bunch of benchmarks
not directly from children
because they haven’t had much time to write very complex programs
but inspired by children, because they like to use big numbers
ran several comparisons

\section{Conclusion}
jumping to arbitrary states is tricky

\section{Distribution of Work}
Regarding division of labor, we have been working collaboratively on Dragon Architect for 14 months. We each implemented several independent optimizations and test cases, and worked together to combine these optimizations into our final algorithm.

\bibliographystyle{plain}
\bibliography{501-final-report}

\end{document}

